\documentclass[a4paper,12pt]{article}
\usepackage[utf8]{inputenc}
\usepackage[english,russian]{babel}
\usepackage{misccorr}
\usepackage{amsmath,amsfonts,amssymb,amsthm,mathtools} 
\usepackage{amsmath}
\usepackage{ dsfont }
\usepackage[argument]{graphicx}
\newcommand{\floor}[1]{\left\lfloor #1 \right\rfloor}
\usepackage{enumitem}
\renewcommand{\headrulewidth}{1.8pt}    % Изменяем размер верхнего отступа колонтитула
\renewcommand{\footrulewidth}{0.0pt}    % Изменяем размер нижнего отступа 
\usepackage[top=0.8in,bottom=0.8in,left=0.8in,right=0.8in,headheight=110pt]{geometry}
\usepackage{fancyhdr}
\pagestyle{fancy}
\fancyhf{}
\fancyfoot[R]{\thepage}
\renewcommand{\headrulewidth}{0pt}
\renewcommand{\footrulewidth}{0pt}
\newcommand{\dlnpn}[1]{\frac{\partial\ln p_n(#1; \theta)}{\partial \theta}}

\begin{document}

\noindent\textbf{Функция распределения}\\

\noindent\textbf{1.} По определению

\begin{center}
$F(x) = \int\limits_{-\infty}^x {1\over{\sum\limits_{i = 0}^k}\sum\limits_{j=0}^k\alpha_i \alpha_j \mathsf{E}X^{i + j}} {e^{-{t - \alpha\over{\beta}}}\over{\beta(1 + e^{-{t - \alpha\over{\beta}}}})^2}(\sum\limits_{i = 0}^k \alpha_i t^i)^2 d t =$
\end{center}

\begin{center}
$= {1\over{\beta\sum\limits_{i = 0}^k}\sum\limits_{j=0}^k\alpha_i \alpha_j \mathsf{E}X^{i + j}} \int\limits_{-\infty}^x \sum\limits_{i = 0}^k \sum\limits_{j = 0}^k \alpha_i \alpha_j t^{i + j} {e^{-{t - \alpha\over{\beta}}}\over{(1 + e^{-{t - \alpha\over{\beta}}}})^2} d t =$
\end{center}

\begin{center}
    $= \mathsf{C_1} \sum\limits_{i = 0}^k \sum\limits_{j = 0}^k \alpha_i \alpha_j \int\limits_{-\infty}^x {1\over{\beta}} t^{i + j} {e^{-{t - \alpha\over{\beta}}}\over{(1 + e^{-{t - \alpha\over{\beta}}}})^2} d t = $
\end{center}

\begin{displaymath}
    = \mathsf{C_1} \begin{pmatrix}
        \alpha_0 & \alpha_1 & \ldots & \alpha_k \\
    \end{pmatrix} \begin{pmatrix}
        \alpha_0 & \alpha_1 & \ldots & \alpha_k & 0 & \ldots & 0 \\
        0 & \alpha_0 & \alpha_1 & \ldots & \alpha_k & \ldots & 0 \\
        \vdots &  & \vdots &  & \vdots & \ddots & \vdots\\
        0 & \ldots & 0 & \alpha_0 & \alpha_1 &\ldots & \alpha_k \\
    \end{pmatrix} \begin{pmatrix}
        \mathsf{I^t_0} \\ \mathsf{I^t_1} \\ \vdots \\ \mathsf{I^t}_{\mathsf{2}k} \\
    \end{pmatrix}
\end{displaymath}\\

\noindent\textbf{2.} Рассмотрим в общем виде интеграл. $l$ определяется параметрами аппроксимации.

\begin{center}
    $\mathsf{I^t}_l = \int\limits_{-\infty}^x {1\over{\beta}} t^l {e^{-{t - \alpha\over{\beta}}}\over{(1 + e^{-{t - \alpha\over{\beta}}}})^2} d t = $
\end{center}

\noindentВведем замену: ${t - \alpha\over{\beta}} = u \Longrightarrow t = \beta u + \alpha$, $y = {x - \alpha\over{\beta}}$

\begin{center}
    $= \int\limits_{-\infty}^y {1\over{\beta}} \beta (\beta u + \alpha)^l {e^{-u}\over{(1 + e^{-u}})^2} d u = \int\limits_{-\infty}^y \sum\limits_{i = 0}^l C_l^i \alpha^{l - i} \beta^i u^i {e^{-u}\over{(1 + e^{-u}})^2} d u = $
\end{center}

\begin{center}
    $= \sum\limits_{i = 0}^l C_l^i \alpha^{l - i} \beta^{i}  \int\limits_{-\infty}^y u^i {e^{-u}\over{(1 + e^{-u}})^2} d u =$
\end{center}

\begin{displaymath}
    = \begin{pmatrix}
        C_l^0 \alpha^l & C_l^1 \alpha^{l - 1} \beta^1 & \ldots & C_l^l \beta^{l} & 0 & \ldots & 0 \\
    \end{pmatrix} \begin{pmatrix}
        \mathsf{I^u_0} \\ \mathsf{I^u_1} \\ \vdots \\ \mathsf{I^u}_{\mathsf{2}k} \\
    \end{pmatrix}
\end{displaymath}

\noindentТогда

\begin{displaymath}
    \begin{pmatrix}
        \mathsf{I^t_0} \\ \mathsf{I^t_1} \\ \vdots \\ \mathsf{I^t}_{\mathsf{2}k} \\
    \end{pmatrix} = \begin{pmatrix}
        C_0^0 \alpha^0 \beta^0 & 0 & \ldots & 0 \\
        C_1^0 \alpha \beta^{0} & C_1^1 \alpha^{0} \beta & \ldots & 0 \\
        \vdots & \vdots & \ddots & \vdots\\
        C_{2k}^0 \alpha^{2k} \beta^0 & C_{2k}^1 \alpha^{2k - 1} \beta & \ldots & C_{2k}^{2k} \alpha^{0} \beta^{2k} \\
    \end{pmatrix} \begin{pmatrix}
        \mathsf{I^u_0} \\ \mathsf{I^u_1} \\ \vdots \\ \mathsf{I^u}_{\mathsf{2}k} \\
    \end{pmatrix}
\end{displaymath}\\

\noindent\textbf{3.} Очередной интеграл, его нахождение и есть основная часть работы. До гиперболического секанса доходим по определению момента логистического распределения как в Вики:

\begin{center}
    $\mathsf{I^u}_m = \int\limits_{-\infty}^y u^m {e^{-u}\over{(1 + e^{-u}})^2} d u =$
\end{center}

\begin{center}
    $= \int\limits_{-\infty}^y u^m {e^{u}e^{-u}\over{e^{u}(1 + 2e^{-u}} + e^{-2u})} d u = \int\limits_{-\infty}^y u^m {1\over{(e^{u} + 2 + e^{-u})}} d u = \int\limits_{-\infty}^y u^m {1\over{4}} {4\over{(e^{u\over{2}} + e^{-{u\over{2}}})^2}} d u =$
\end{center}

\begin{center}
    $= {1\over{4}}\int\limits_{-\infty}^y u^m \operatorname{sech}^2{u\over{2}} d u$
\end{center}

\noindentФункция симметрична относительно начала координат, если $m$ нечетно, симметрична относительно оси $O y$, если $m$ четно. Тут уже  

\begin{center}
    $\mathsf{I^u}_m = (-1)^m{1\over{4}} \int\limits_0^{\infty} u^m \operatorname{sech}^2{u\over{2}} d u  + (-1)^{[m \equiv 0 \pmod{2}][y < 0]}{1\over{4}}  \int\limits_0^{|y|} u^m \operatorname{sech}^2{u\over{2}} d u = $
\end{center}

\noindentПервое слагаемое – считается легко, зависит от только от порядка (порядком считаю степень переменной, которая определяется параметрами аппроксимации). Нашла в статье (1), ссылку прикреплю позже. Второе слагаемое меняется в зависимости от аргумента функции распределения.\\

\noindentОбозначим $\mathsf{I^{v}}_m = \int\limits_0^{v} u^m \operatorname{sech}^2{u\over{2}} d u$, 

\begin{center}
    $= (-1)^m {1\over{4}} \mathsf{I^{v = inf}}_m  + (-1)^{[m \equiv 0 \pmod{2}][y < 0]} {1\over{4}} \mathsf{I^{v = |y|}}_m$
\end{center}

\noindentРассмотрим два частных случая (это действительно важно, тут есть слагаемые, которые с ростом порядка исчезают, общую формулу вывести трудно, только если индикаторы добавить):\\

\noindent\textbf{4.} Пусть m = 0

\begin{center}
    $\mathsf{I^{v}}_0 = \int\limits_0^{v} \operatorname{sech}^2{u\over{2}} d u = 2\int\limits_{0}^v d\operatorname{tanh}{u\over{2}} = 2 \operatorname{tanh}{v\over{2}}$
\end{center}

\noindent\textbf{5.} Пусть m = 1

\begin{center}
    $\mathsf{I^{v}}_1 = \int\limits_0^{v} u \operatorname{sech}^2{u\over{2}} d u =$
\end{center}

\begin{center}
    $= 2\int\limits_{0}^v u d\operatorname{tanh}{u\over{2}} = 2v \operatorname{tanh}{v\over{2}} - 4\int\limits_0^v {1\over{2}} \operatorname{tanh}{v\over{2}}du = 2v\operatorname{tanh}{v\over{2}} - 4\ln{\operatorname{cosh}{v\over{2}}} =$
\end{center}

\begin{center}
    $= 2v\operatorname{tanh}{v\over{2}} - 2v + 4\operatorname{Li_1}(-e^{-v}) + 4\ln{2}$
\end{center}\\

\noindentДальше до пункта 7 нет шагов сложнее интегралов по частям и использования определения и базовых свойств полилогарифма, которые легко гуглятся и есть в вольфраме, добавлю позже.\\

\noindent\textbf{6.} Пусть $m \geq{2}$

\begin{center}
    $\mathsf{I^{v}}_m = \int\limits_0^{v} u^m \operatorname{sech}^2{u\over{2}} d u =$
\end{center}

\begin{center}
    $= 2\int\limits_{0}^v u^m d\operatorname{tanh}{u\over{2}} = 2v^m \operatorname{tanh}{v\over{2}} - 2\int\limits_0^v m u^{m - 1} \operatorname{tanh}{u\over{2}}d u = $
\end{center}

\begin{center}
    $= 2v^m \operatorname{tanh}{v\over{2}} - 4 m \int\limits_0^v u^{m - 1} d \ln{\operatorname{cosh}{u\over{2}}} =$
\end{center}

\begin{center}
    $= 2v^m \operatorname{tanh}{v\over{2}} - 4 m v^{m - 1} \ln{\operatorname{cosh}{v\over{2}}} + 4m\int\limits_0^v (m - 1) u^{m - 2} \ln{\operatorname{cosh}{u\over{2}}} d u  =$
\end{center}

\begin{center}
    $= 2v^m \operatorname{tanh}{v\over{2}} - 4 m v^{m - 1}({v\over{2}} - \operatorname{Li_1}(-e^{-v}) - \ln{2}) + 4m(m - 1) \int\limits_0^v u^{m - 2} ({u\over{2}} - \operatorname{Li_1}(-e^{-u}) - \ln{2}) d u  =$
\end{center}

\begin{center}
    $= 2v^m \operatorname{tanh}{v\over{2}} - 2v^m + 4m v^{m - 1}\operatorname{Li_1}(-e^{-v}) - 4m(m - 1) \int\limits_0^v u^{m - 2} \operatorname{Li_1}(-e^{-u}) d u  =$
\end{center}

\begin{center}
    $= 2v^m \operatorname{tanh}{v\over{2}} - 2v^m + 4m v^{m - 1}\operatorname{Li_1}(-e^{-v}) - 4m(m - 1) \int\limits_0^v \operatorname{Li_1}(-e^{-u}) d {u^{m - 1}\over{m - 1}}  =$
\end{center}

\begin{center}
    $= 2v^m \operatorname{tanh}{v\over{2}} - 2v^m + 4m v^{m - 1}\operatorname{Li_1}(-e^{-v}) - 4m u^{m - 1}\operatorname{Li_1}(-e^{-u}) - 4m(m - 1)\int\limits_0^v {u^{m - 1}\over{m - 1}} d\ln(1 + e^{-u}) =$
\end{center}

\begin{center}
    $= 2v^m \operatorname{tanh}{v\over{2}} - 2v^m + 4m\int\limits_0^v u^{m - 1} {1\over{1 + e^{u}}}d u =$
\end{center}

\noindentОбозначим $\mathsf{I^f}_m = \int\limits_0^v u^{m} {1\over{1 + e^{u}}}d u$

\begin{center}
    $= 2v^m \operatorname{tanh}{v\over{2}} - 2v^m + 4m \mathsf{I^f}_{m - 1}$
\end{center}

\noindent\textbf{7.} Самое веселое начинается здесь. Переход между второй и третьей строкой надо пояснить формулой из статьи (2).

\begin{center}
    $\mathsf{I^f}_m = \int\limits_0^v u^{m} {1\over{1 + e^{u}}}d u =$
\end{center}

\begin{center}
    $= \int\limits_0^{\infty} u^{m} {1\over{1 + e^{u}}}d u - \int\limits_v^{\infty} u^{m} {1\over{1 + e^{u}}}d u =$
\end{center}

\begin{center}
    $= \Gamma(m + 1)(1 - 2^{- m})\zeta(m + 1) - \int\limits_v^{\infty} u^{m} {1\over{1 + e^{u}}}d u = $
\end{center}

\begin{center}
    $= \Gamma(m + 1)(1 - 2^{- m})\zeta(m + 1) - \int\limits_0^{\infty} (v + u)^{m} {1\over{1 + e^{v + u}}}d u = $
\end{center}

\begin{center}
    $= \Gamma(m + 1)(1 - 2^{- m})\zeta(m + 1) - \sum\limits_{i = 0}^m C_m^i v^{m - i} \int\limits_0^{\infty} u^{i} {1\over{1 + e^{v + u}}}d u = $
\end{center}

\noindentЗаметим, что ${1\over{1 + e^{u}}} = \sum\limits_{j = 1}^{\infty} (-1)^{j - 1}e^{-j u}$

\begin{center}
    $= \Gamma(m + 1)(1 - 2^{- m})\zeta(m + 1) - \sum\limits_{i = 0}^m C_m^i v^{m - i} \int\limits_0^{\infty} u^{i} \sum\limits_{j = 1}^{\infty} (-1)^{j - 1}e^{-j(v + u)}d u = $
\end{center}

\begin{center}
    $= \Gamma(m + 1)(1 - 2^{- m})\zeta(m + 1) - \sum\limits_{i = 0}^m C_m^i v^{m - i} \sum\limits_{j = 1}^{\infty} (-1)^{j - 1}e^{-j v}\int\limits_0^{\infty} u^{i} e^{-j u}d u = $
\end{center}

\begin{center}
    $= \Gamma(m + 1)(1 - 2^{- m})\zeta(m + 1) - \sum\limits_{i = 0}^m C_m^i v^{m - i} \sum\limits_{j = 1}^{\infty} (-1) (-1)^{j} (e^{-v})^j\int\limits_0^{\infty}{1\over{j^{i + 1}}} (j u)^{i} e^{-j u}d (j u) = $
\end{center}

\noindentПоскольку $\int\limits_0^{\infty} u^{i} e^{-u}d u = \Gamma(i + 1)$

\begin{center}
    $= \Gamma(m + 1)(1 - 2^{- m})\zeta(m + 1) + \sum\limits_{i = 0}^m {\Gamma(m + 1)\over{\Gamma(i + 1) \Gamma(m - i + 1)}} v^{m - i} \sum\limits_{j = 1}^{\infty} {(-e^{-v})^j\over{j^{i + 1}}} \Gamma(i + 1) = $
\end{center}

\noindentПо определению $\operatorname{Li_p(z)} = \sum\limits_{j = 1}^{\infty} {z^{j}\over{j^p}}$

\begin{center}
    $= \Gamma(m + 1)(1 - 2^{- m})\zeta(m + 1) + \Gamma(m + 1)\sum\limits_{i = 0}^m {v^{m - i}\over{\Gamma(m - i + 1)}} \operatorname{Li_{i + 1}}(-e^{-v})$
\end{center}\\

\noindentНайдем несколько констант.\\

\noindent\textbf{8.} Если $v \longrightarrow \infty$, то

\begin{center}
    $\mathsf{I^{v}}_m = \int\limits_0^{v} u^m \operatorname{sech}^2{u\over{2}} d u$
\end{center}

\begin{center}
    $\lim\limits_{v \rightarrow \infty} {1\over{4}}\mathsf{I^{v}}_0 = \lim\limits_{v \rightarrow \infty} {1\over{2}} \operatorname{tanh}{v\over{2}} = {1\over{2}}$
\end{center}

\begin{center}
    $\lim\limits_{v \rightarrow \infty} {1\over{4}} \mathsf{I^{v}}_1 = \lim\limits_{v \rightarrow \infty} ({1\over{2}} v\operatorname{tanh}{v\over{2}} - {1\over{2}}v + \operatorname{Li_1}(-e^{-v}) + \ln{2}) = \ln{2}$
\end{center}

\begin{center}
    $\lim\limits_{v \rightarrow \infty} {1\over{4}} \mathsf{I^{v}}_m = \lim\limits_{v \rightarrow \infty} ({1\over{2}} v^m \operatorname{tanh}{v\over{2}} - {1\over{2}}v^m) + m \Gamma(m)(1 - 2^{1 - m})\zeta(m) = m \Gamma(m)(1 - 2^{1 - m})\zeta(m)$
\end{center}

\noindent\textbf{9.} Сведем все к матрицам. Обозначим $\mathsf{C_m} = m \Gamma(m)(1 - 2^{1 - m})\zeta(m)$

\begin{center}
    $\mathsf{I^u}_m = (-1)^m \mathsf{C_m} + (-1)^{[m \equiv 0 \pmod{2}][y < 0]} \left({1\over{2}}|y|^m \operatorname{tanh}{|y|\over{2}} - {1\over{2}}|y|^m + \mathsf{C_m} + m\Gamma(m)\sum\limits_{i = 0}^{m - 1} {|y|^{m - i - 1}\over{\Gamma(m - i)}} \operatorname{Li_{i + 1}}(-e^{-|y|})\right) =$
\end{center}

\begin{center}
    $= \mathsf{C_m} \left((-1)^m + (-1)^{[m \equiv 0 \pmod{2}][y < 0]}\right)  + (-1)^{[m \equiv 0 \pmod{2}][y < 0]} \left({1\over{2}}|y|^m (\operatorname{tanh}{|y|\over{2}} - 1) + m \Gamma(m)\sum\limits_{i = 0}^{m - 1} {|y|^{m - i - 1}\over{\Gamma(m - i)}} \operatorname{Li_{i + 1}}(-e^{-|y|})\right) =$
\end{center}

\noindentОбозначим $\mathsf{C_m^{y}} = (-1)^{[m \equiv 0 \pmod{2}][y < 0]}$, $\mathsf{C_m^{inf}} = (-1)^m$

\begin{center}
    $= \mathsf{C_m} (\mathsf{C_m^{inf}}+ \mathsf{C_m^{y}})  + \mathsf{C_m^{y}} \left({1\over{2}}|y|^m (\operatorname{tanh}{|y|\over{2}} - 1) + m \Gamma(m)\sum\limits_{i = 0}^{m - 1} {|y|^{m - i - 1}\over{\Gamma(m - i)}} \operatorname{Li_{i + 1}}(-e^{-|y|})\right)$
\end{center}

\begin{center}
    $\mathsf{I^u}_0 = \lim\limits_{y\rightarrow \infty} {1\over{2}}|y|^m \operatorname{tanh}{|y|\over{2}} + \mathsf{C_m^{y}}  {1\over{2}}|y|^m \operatorname{tanh}{|y|\over{2}}$
\end{center}

\begin{displaymath}
    \begin{pmatrix}
        \mathsf{I^u_1} \\ \mathsf{I^u_2} \\ \vdots \\ \mathsf{I^u}_{\mathsf{2}k} \\
    \end{pmatrix} = \mathsf{C} \circ (\mathsf{C^{inf}} + \mathsf{C^y}) + \mathsf{C^y} \circ \left({1\over{2}}\operatorname{tanh}{|y|\over{2}} - {1\over{2}}\right) \begin{pmatrix}
        |y|^1 \\ |y|^2 \\ \vdots \\ |y|^{\mathsf{2}k} \\
    \end{pmatrix} + \mathsf{C^y} \circ \begin{pmatrix}
        \Gamma(1) \\ 2\Gamma(2) \\ \vdots \\ 2k\Gamma(2k) \\
    \end{pmatrix} \circ \mathsf{C^{Li}} 
\end{displaymath}

\noindentГде $\circ$ – символ поэлементного умножения,

\begin{displaymath}
    \mathsf{C^{Li}} =
    \begin{pmatrix}
        1\over{\Gamma(1)} & 0 & \ldots & 0 \\
        |y|\over{\Gamma(2)} & 1\over{\Gamma(1)} & \ldots & 0 \\
        \vdots & \vdots & \ddots & 0 \\
        |y|^{2k - 1}\over{\Gamma(2k)} & |y|^{2k - 2}\over{\Gamma(2k - 1)} & \ldots & 1\over{\Gamma(1)} \\
    \end{pmatrix} \begin{pmatrix}
        \operatorname{Li_1}(-e^{-|y|}) \\ \operatorname{Li_2}(-e^{-|y|}) \\ \vdots \\ \operatorname{Li_{2k}}(-e^{-|y|}) \\
    \end{pmatrix} 
\end{displaymath}\\

\noindentПобочное: посчитаются моменты, очев лучше, чем в сайпае. Для оптимизации понадобятся градиенты, все уже найдено, градиенты перепроверить численно.\\

\noindent\textbf{10.} Моменты

\begin{center}
    ${\sum\limits_{i = 0}^k}\sum\limits_{j=0}^k\alpha_i \alpha_j \mathsf{E}X^{i + j} = $
\end{center}

\begin{displaymath}
    = \begin{pmatrix}
        \alpha_0 & \alpha_1 & \ldots & \alpha_k \\
    \end{pmatrix} \begin{pmatrix}
        \alpha_0 & \alpha_1 & \ldots & \alpha_k & 0 & \ldots & 0 \\
        0 & \alpha_0 & \alpha_1 & \ldots & \alpha_k & \ldots & 0 \\
        \vdots &  & \vdots &  & \vdots & \ddots & \vdots\\
        0 & \ldots & 0 & \alpha_0 & \alpha_1 &\ldots & \alpha_k \\
    \end{pmatrix} \begin{pmatrix}
        \mathsf{E}X^{\mathsf{0}} \\ \mathsf{E}X^\mathsf{1} \\ \vdots \\ \mathsf{E}X^{\mathsf{2}k} \\
    \end{pmatrix}
\end{displaymath}

\begin{center}
    $\mathsf{E}X^{q} = \int\limits_{-\infty}^{\infty} x^{q} {e^{-{x - \alpha\over{\beta}}}\over{\beta(1 + e^{-{x - \alpha\over{\beta}}}})^2} d x$
\end{center}

\noindentВведем замену: ${x - \alpha\over{\beta}} = y \Longrightarrow x = \beta y + \alpha$

\begin{center}
    $= \int\limits_{-\infty}^{\infty} {1\over{\beta}} \beta (\beta y + \alpha)^q {e^{-y}\over{(1 + e^{-y}})^2} d y = \int\limits_{-\infty}^{\infty} \sum\limits_{i = 0}^q C_q^i \alpha^{q - i} \beta^i y^i {e^{-y}\over{(1 + e^{-y}})^2} d y = $
\end{center}

\begin{center}
    $= \sum\limits_{i = 0}^q C_q^i \alpha^{q - i} \beta^{i}  \int\limits_{-\infty}^{\infty} y^i {e^{-y}\over{(1 + e^{-y}})^2} d y =$
\end{center}

\begin{displaymath}
    = \begin{pmatrix}
        C_q^0 \alpha^q & C_q^1 \alpha^{q - 1} \beta^1 & \ldots & C_q^q \beta^{q} & 0 & \ldots & 0 \\
    \end{pmatrix} \begin{pmatrix}
        {1\over{2}} + (-1)^{0}{1\over{2}} \\ \ln{2} + (-1)^{1}\ln{2} \\ \vdots \\ 2k \Gamma(2k) (1 - 2^{2k}) \zeta(2k) + (-1)^{2k} 2k \Gamma(2k) (1 - 2^{2k}) \zeta(2k)\\
    \end{pmatrix}
\end{displaymath}

\noindent\textbf{11.} Градиент

\begin{center}
    $L(\alpha, w; x, y) = -\sum\limits_{i = 0}^l y_i \ln{F(\alpha, \langle w, x_i\rangle)} + (1 - y_i) \ln{(1 - F(\alpha, \langle w, x_i \rangle))}$
\end{center}

\noindentПусть $\theta = (\alpha, w)$, обозначим $\mathsf{F_i} = F(\alpha, \langle w, x_i\rangle)$

\begin{center}
    ${\partial\over{\partial \theta}} L(\alpha, w; x, y) = -\sum\limits_{i = 0}^l {\partial\over{\partial \theta}} \left( y_i \ln{\mathsf{F_i}} + (1 - y_i) \ln{(1 - \mathsf{F_i})} \right) =$
\end{center}

\begin{center}
    $= \sum\limits_{i = 0}^l {-y_i\over{\mathsf{F_i}}} {\partial\over{\partial \theta}} \mathsf{F_i} + {1 - y_i\over{1 - \mathsf{F_i}}} {\partial\over{\partial \theta}} \mathsf{F_i} = \sum\limits_{i = 0}^l {-y_i + y_i \mathsf{F_i} + \mathsf{F_i} - y_i \mathsf{F_i}\over{\mathsf{F_i} (1 - \mathsf{F_i})}} {\partial\over{\partial \theta}} \mathsf{F_i} =$
\end{center}

\begin{center}
    $= \sum\limits_{i = 0}^l {\mathsf{F_i} -y_i\over{\mathsf{F_i} (1 - \mathsf{F_i})}} {\partial\over{\partial \theta}} \mathsf{F_i} = \sum\limits_{i = 0}^l {\mathsf{F_i} -y_i\over{\mathsf{F_i} (1 - \mathsf{F_i})}} ({\partial\over{\partial \alpha}} \mathsf{F_i}\,, {\partial\over{\partial w}} \mathsf{F_i})$
\end{center}

\noindent\textbf{12.}

\begin{center}
    ${\partial\over{\partial \alpha}} \mathsf{F_i} = {\partial\over{\partial \alpha}} {1\over{\sum\limits_{i = 0}^k}\sum\limits_{j=0}^k\alpha_i \alpha_j \mathsf{E}X^{i + j}} \sum\limits_{i = 0}^k \sum\limits_{j = 0}^k \alpha_i \alpha_j \int\limits_{-\infty}^{\langle w \,, x_{\mathsf{i}}\rangle} {1\over{\beta}} t^{i + j} {e^{-{t - \alpha\over{\beta}}}\over{(1 + e^{-{t - \alpha\over{\beta}}}})^2} d t = $
\end{center}

\begin{center}
    $= {\partial\over{\partial \alpha}} {\sum\limits_{i = 0}^k \sum\limits_{j = 0}^k \alpha_i \alpha_j \mathsf{I^{ti}}_{i + j}\over{\sum\limits_{i = 0}^k}\sum\limits_{j=0}^k\alpha_i \alpha_j \mathsf{E}X^{i + j}} $
\end{center}

\begin{displaymath}
    {\partial\over{\partial \alpha}} \sum\limits_{i = 0}^k \sum\limits_{j = 0}^k \alpha_i \alpha_j \mathsf{I^{t}}_{i + j} = (2 \sum\limits_{j = 0}^k \alpha_j \mathsf{I^{t}}_{i + j})_{i = 0}^k
    = 2\begin{pmatrix}
        \alpha_0 & \alpha_1 & \ldots & \alpha_k & 0 & \ldots & 0 \\
        0 & \alpha_0 & \alpha_1 & \ldots & \alpha_k & \ldots & 0 \\
        \vdots &  & \vdots &  & \vdots & \ddots & \vdots\\
        0 & \ldots & 0 & \alpha_0 & \alpha_1 &\ldots & \alpha_k \\
    \end{pmatrix} \begin{pmatrix}
        \mathsf{I^t_0} \\ \mathsf{I^t_1} \\ \vdots \\ \mathsf{I^t}_{\mathsf{2}k} \\
    \end{pmatrix} = \mathsf{A}
\end{displaymath}

\begin{displaymath}
    {\partial\over{\partial \alpha}} {\sum\limits_{i = 0}^k}\sum\limits_{j=0}^k\alpha_i \alpha_j \mathsf{E}X^{i + j} = (2\sum\limits_{j = 0}^k \alpha_j \mathsf{E}X^{i + j})_{i = 0}^k
    = \begin{pmatrix}
        \alpha_0 & \alpha_1 & \ldots & \alpha_k & 0 & \ldots & 0 \\
        0 & \alpha_0 & \alpha_1 & \ldots & \alpha_k & \ldots & 0 \\
        \vdots &  & \vdots &  & \vdots & \ddots & \vdots\\
        0 & \ldots & 0 & \alpha_0 & \alpha_1 &\ldots & \alpha_k \\
    \end{pmatrix} \begin{pmatrix}
        \mathsf{E}X^{\mathsf{0}} \\ \mathsf{E}X^\mathsf{1} \\ \vdots \\ \mathsf{E}X^{\mathsf{2}k} \\
    \end{pmatrix} = \mathsf{E}
\end{displaymath}

\begin{center}
    ${\partial\over{\partial \alpha}} \mathsf{F_i} = {\mathsf{A}\mathsf{C_1}} - {\mathsf{E}\mathsf{C_1}^2 \sum\limits_{i = 0}^k \sum\limits_{j = 0}^k \alpha_i \alpha_j \mathsf{I^{t}}_{i + j}} =$
\end{center}

\begin{displaymath}
    = {\mathsf{A}\mathsf{C_1}} - {\mathsf{E}\mathsf{C_1}^2} \begin{pmatrix}
        \alpha_0 & \alpha_1 & \ldots & \alpha_k \\
    \end{pmatrix} \begin{pmatrix}
        \alpha_0 & \alpha_1 & \ldots & \alpha_k & 0 & \ldots & 0 \\
        0 & \alpha_0 & \alpha_1 & \ldots & \alpha_k & \ldots & 0 \\
        \vdots &  & \vdots &  & \vdots & \ddots & \vdots\\
        0 & \ldots & 0 & \alpha_0 & \alpha_1 &\ldots & \alpha_k \\
    \end{pmatrix} \begin{pmatrix}
        \mathsf{I^t_0} \\ \mathsf{I^t_1} \\ \vdots \\ \mathsf{I^t}_{\mathsf{2}k} \\
    \end{pmatrix}
\end{displaymath}\\

\noindent\textbf{13.} Для случая $\langle w \,, x_{\mathsf{i}}\rangle \geq{0}$

\begin{center}
    ${\partial\over{\partial w}} \mathsf{F_i} = {\partial\over{\partial w}} \mathsf{C_1} \sum\limits_{i = 0}^k \sum\limits_{j = 0}^k \alpha_i \alpha_j \int\limits_{-\infty}^{\langle w \,, x_{\mathsf{i}}\rangle} {1\over{\beta}} t^{i + j} {e^{-{t - \alpha\over{\beta}}}\over{(1 + e^{-{t - \alpha\over{\beta}}}})^2} d t = $
\end{center}

\begin{center}
    $= \mathsf{C_1} \sum\limits_{i = 0}^k \sum\limits_{j = 0}^k \alpha_i \alpha_j {\partial\over{\partial w}} \int\limits_{-\infty}^{\langle w \,, x_{\mathsf{i}}\rangle} {1\over{\beta}} t^{i + j} {e^{-{t - \alpha\over{\beta}}}\over{(1 + e^{-{t - \alpha\over{\beta}}}})^2} d t = $
\end{center}

\begin{center}
    $= \mathsf{C_1} \sum\limits_{i = 0}^k \sum\limits_{j = 0}^k \alpha_i \alpha_j  {1\over{\beta}} \langle w \,, x_{\mathsf{i}}\rangle^{i + j} {e^{-{\langle w \,, x_{\mathsf{i}}\rangle - \alpha\over{\beta}}}\over{\left(1 + e^{-{\langle w \,, x_{\mathsf{i}}\rangle - \alpha\over{\beta}}}\right)^2}} {\partial\over{\partial w}} \langle w \,, x_{\mathsf{i}}\rangle = $
\end{center}

\begin{center}
    $= {e^{-{\langle w \,, x_{\mathsf{i}}\rangle - \alpha\over{\beta}}}\over{\beta\left(1 + e^{-{\langle w \,, x_{\mathsf{i}}\rangle - \alpha\over{\beta}}}\right)^2}} x_{\mathsf{i}} \mathsf{C_1} \sum\limits_{i = 0}^k \sum\limits_{j = 0}^k \alpha_i \alpha_j \langle w \,, x_{\mathsf{i}}\rangle^{i + j}$
\end{center}

\begin{displaymath}
    {\partial\over{\partial w}} \mathsf{F_i} = {e^{-{\langle w \,, x_{\mathsf{i}}\rangle - \alpha\over{\beta}}}\over{\beta\left(1 + e^{-{\langle w \,, x_{\mathsf{i}}\rangle - \alpha\over{\beta}}}\right)^2}} x_{\mathsf{i}} \mathsf{C_1} \begin{pmatrix}
        \alpha_0 & \alpha_1 & \ldots & \alpha_k \\
    \end{pmatrix} \begin{pmatrix}
        \alpha_0 & \alpha_1 & \ldots & \alpha_k & 0 & \ldots & 0 \\
        0 & \alpha_0 & \alpha_1 & \ldots & \alpha_k & \ldots & 0 \\
        \vdots &  & \vdots &  & \vdots & \ddots & \vdots\\
        0 & \ldots & 0 & \alpha_0 & \alpha_1 &\ldots & \alpha_k \\
    \end{pmatrix} \begin{pmatrix}
        \langle w \,, x_{\mathsf{i}}\rangle^{0} \\ \langle w \,, x_{\mathsf{i}}\rangle^{1} \\ \vdots \\ \langle w \,, x_{\mathsf{i}}\rangle^{\mathsf{2}k} \\
    \end{pmatrix}
\end{displaymath}\\

\noindent\textbf{14.} Для случая $\langle w \,, x_{\mathsf{i}}\rangle < 0$

\begin{center}
    ${\partial\over{\partial w}} \mathsf{F_i} = {\partial\over{\partial w}} \mathsf{C_1} \sum\limits_{i = 0}^k \sum\limits_{j = 0}^k \alpha_i \alpha_j \int\limits_{-\infty}^{\langle w \,, x_{\mathsf{i}}\rangle} {1\over{\beta}} t^{i + j} {e^{-{t - \alpha\over{\beta}}}\over{(1 + e^{-{t - \alpha\over{\beta}}}})^2} d t = $
\end{center}

\begin{center}
    $= \mathsf{C_1} \sum\limits_{i = 0}^k \sum\limits_{j = 0}^k \alpha_i \alpha_j {\partial\over{\partial w}} \int\limits_{-\infty}^{\langle w \,, x_{\mathsf{i}}\rangle} {1\over{\beta}} t^{i + j} {e^{-{t - \alpha\over{\beta}}}\over{(1 + e^{-{t - \alpha\over{\beta}}}})^2} d t$
\end{center}

\noindentЕсли $i + j$ четно, то функция симметрична относительно оси ординат.

\begin{center}
    $\int\limits_{-\infty}^{\langle w \,, x_{\mathsf{i}}\rangle} {1\over{\beta}} t^{i + j} {e^{-{t - \alpha\over{\beta}}}\over{(1 + e^{-{t - \alpha\over{\beta}}}})^2} d t =
    \int\limits_{-\infty}^{0} {1\over{\beta}} t^{i + j} {e^{-{t - \alpha\over{\beta}}}\over{(1 + e^{-{t - \alpha\over{\beta}}}})^2} d - \int\limits_{0}^{\langle w \,, x_{\mathsf{i}}\rangle} {1\over{\beta}} t^{i + j} {e^{-{t - \alpha\over{\beta}}}\over{(1 + e^{-{t - \alpha\over{\beta}}}})^2} d t$
\end{center}

\begin{center}
    ${\partial\over{\partial w}}\int\limits_{-\infty}^{\langle w \,, x_{\mathsf{i}}\rangle} {1\over{\beta}} t^{i + j} {e^{-{t - \alpha\over{\beta}}}\over{(1 + e^{-{t - \alpha\over{\beta}}}})^2} d t = - {\partial\over{\partial w}}\int\limits_{0}^{\langle w \,, x_{\mathsf{i}}\rangle} {1\over{\beta}} t^{i + j} {e^{-{t - \alpha\over{\beta}}}\over{(1 + e^{-{t - \alpha\over{\beta}}}})^2} d t = -{1\over{\beta}}x_{\mathsf{i}}\langle w \,, x_{\mathsf{i}}\rangle^{i + j} {e^{-{\langle w \,, x_{\mathsf{i}}\rangle - \alpha\over{\beta}}}\over{\left(1 + e^{-{\langle w \,, x_{\mathsf{i}}\rangle - \alpha\over{\beta}}}\right)^2}}$
\end{center}

\noindentЕсли $i + j$ нечетно, то это соответствует ситуации $\langle w \,, x_{\mathsf{i}}\rangle \geq{0}$. Тогда

\begin{center}
    ${\partial\over{\partial w}} \mathsf{F_i} = {e^{-{\langle w \,, x_{\mathsf{i}}\rangle - \alpha\over{\beta}}}\over{\beta\left(1 + e^{-{\langle w \,, x_{\mathsf{i}}\rangle - \alpha\over{\beta}}}\right)^2}}x_{\mathsf{i}}\mathsf{C_1} \sum\limits_{i = 0}^k \sum\limits_{j = 0}^k \alpha_i \alpha_j  (-1)^{[(i + j) \equiv 0 \pmod{2}][\langle w \,, x_{\mathsf{i}}\rangle < 0]} \langle w \,, x_{\mathsf{i}}\rangle^{i + j}$
\end{center}

\noindent\textbf{15.}В конечном итоге имеем\\

\noindentОбозначим $\mathsf{C^{\langle w \,, x_{\mathsf{i}}\rangle}_{i + j}} = (-1)^{[(i + j) \equiv 0 \pmod{2}][\langle w \,, x_{\mathsf{i}}\rangle < 0]}$

\begin{displaymath}
    {\partial\over{\partial w}} \mathsf{F_i} = {e^{-{\langle w \,, x_{\mathsf{i}}\rangle - \alpha\over{\beta}}}\over{\beta\left(1 + e^{-{\langle w \,, x_{\mathsf{i}}\rangle - \alpha\over{\beta}}}\right)^2}} x_{\mathsf{i}} \mathsf{C_1} \begin{pmatrix}
        \alpha_0 & \alpha_1 & \ldots & \alpha_k \\
    \end{pmatrix} \begin{pmatrix}
        \alpha_0 & \alpha_1 & \ldots & \alpha_k & 0 & \ldots & 0 \\
        0 & \alpha_0 & \alpha_1 & \ldots & \alpha_k & \ldots & 0 \\
        \vdots &  & \vdots &  & \vdots & \ddots & \vdots\\
        0 & \ldots & 0 & \alpha_0 & \alpha_1 &\ldots & \alpha_k \\
    \end{pmatrix} \mathsf{C^{\langle w \,, x_{\mathsf{i}}\rangle}}\circ \begin{pmatrix}
        \langle w \,, x_{\mathsf{i}}\rangle^{0} \\ \langle w \,, x_{\mathsf{i}}\rangle^{1} \\ \vdots \\ \langle w \,, x_{\mathsf{i}}\rangle^{\mathsf{2}k} \\
    \end{pmatrix}
\end{displaymath}\\

\end{document}